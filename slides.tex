\documentclass[pdf]{beamer}
\mode<presentation>{}
%\usetheme{Rochester}
\usecolortheme{orchid}
%\usecolortheme{seahorse}
\setbeamertemplate{theorems}[ams style]

\usepackage[sans]{dsfont}
\usepackage{amsmath}
\usepackage{csquotes}
\usepackage[all,cmtip]{xy}

\newcommand{\op}[1]{\operatorname{#1}}
\newcommand{\bbf}[1]{\mathds{#1}}
\newcommand{\Z}{\bbf{Z}}
\newcommand{\F}{\bbf{F}}
\newcommand{\C}{\bbf{C}}
\newcommand{\X}{\bbf{X}}

\newtheorem{question}{Question}
\newtheorem*{theorem*}{Theorem}
\newtheorem*{lemma*}{Lemma}
\newtheorem*{cor*}{Corollary}

\title{The Cohomology of Root Data}
\author{Nicolas A. Schmidt}
\date{8 October 2018}

\begin{document}

%% title frame
\begin{frame}
   \titlepage
\end{frame}

\begin{frame}{What's an ``automorphic technique''?}
   \pause
   According to \textit{Kurt Eissler}:
  \pause
   \blockquote[]{
      \textins{\textit{Automorphic techniques} are} artistic activites which take place in a borderland of autoplastic and alloplastic attitudes toward reality.\uncover<4->{ A work of art \textelp{} is \textit{autoplastic} in so far as, like a dream or symptom, it serves the {\color<6->[rgb]{1,0,0}solution of an inner conflict} and the {\color<6->[rgb]{1,0,0}fulfillment of a wish};\uncover<5->{ it is simultaneously \textit{alloplastic}, however, since it modifies reality by the {\color<6->[rgb]{1,0,0}creation of something original and new.}}}
   }
   (Taken from H. Kohut, \textit{Forms and Transformations of Narcicissm.} American Imago, 1967)
\end{frame}

\begin{frame}{The problem}
   \pause{}Given
   \begin{itemize}
      \item $G$ - split reductive group $G$ over field $k$\pause
      \item $T$ - maximal split torus in $G$
   \end{itemize}
   \pause have extension
   \[ \xymatrix{ 1 \ar[r] & T(k) \ar[r] & N(k) \ar[r] & W \ar[r] & 1 } \]
   \pause where
   \begin{itemize}
      \item $N$ - normalizer of $T$\pause
      \item $W$ - Weyl group\pause{}, subgroup of $\op{Aut}(T) = \op{GL}_{\Z}(X_\ast(T))$
   \end{itemize}\pause
   \begin{question}
      When does this extension split?
   \end{question}
\end{frame}

\begin{frame}{Brief history of the problem}
   \pause\begin{itemize}
      \item[1966] Tits' article ``Normalisateurs de Tores I.'' appears. \pause Announces complete solution in ``Normalisateurs de Tores II.''\pause , which \textit{never appeared}.\pause 
      \item<-8>[1974] Curtis, Wiederhold, and Williams give answer for \textit{simple} groups over $k = \C$.\pause
      \item<-8>[2013--17] Gal't solves it case-by-case for $k = \overline{\F}_p$ in several articles.\pause
      \item<-8>[2017] Question is settled in general by Adams and He.\pause
      \item[2018] I attempt myself at it.\pause
   \end{itemize}
\end{frame}

\begin{frame}{\dots and how I came to it}
   \pause For every Coxeter group $(W,S)$, there exists a canonical $2$-cocycle\pause
   \[ \X: W \times W \longrightarrow \Z[\mathfrak{H}] \]
   \pause given by
   \[ \X(w,w') = \prod_{ H \in \mathfrak{H}(w,w')} a_H \]
   \pause where
   \begin{align*} \Z[\mathfrak{H}] & = \text{free (multiplicative) abelian group on symbols $a_H$} \\
      \uncover<6->{\mathfrak{H} & = \{ wsw^{-1} : w \in W,\ s \in S\} \pause \uncover<7->{= \text{``hyperplanes''}} \\}
      \uncover<8->{\mathfrak{H}(w,w') & = \{ H \in \mathfrak{H} : \text{\uncover<9->{$H$ separates both $1$ from $w$ and $w$ from $ww'$}}\}\\}
      \uncover<10->{ & = \text{hyperplanes crossed twice by $1-w-ww'$}\\}
      \uncover<11->{ & = \text{\textbf{the thing that makes your life difficult}}}
   \end{align*}
\end{frame}

\begin{frame}{\dots and how I came to it}
   \pause Given $(W,S)$, consider category $\mathcal{C}(W,S)$ with\pause
\begin{itemize}
   \item[objs] Extensions \uncover<4->{together with set-theoretic sections $n$}
      \[ \xymatrix{ 1 \ar[r] & T \ar[r] & W(1) \ar[r] & W \ar[r] \alt<4->{\ar@/_1.5em/[l]_n}{\ar@[white]@/_1.5em/[l]_{\color{white}n}} & 1 } \]
   \pause[5]satisfying the ``braid relations''
   \[ \ell(ww') = \ell(w)+\ell(w') \quad \Rightarrow \quad n(ww') = n(w)n(w') \]
\item<6->[morphs] Commutative diagrams\uncover<7->{ preserving sections \uncover<8->{($\phi \circ n = n'$)}}
      \[ \xymatrix{ 1 \ar[r] & T \ar[d] \ar[r] & W(1) \ar[d]^\phi \ar[r] & W \ar[d]^{\op{id}} \ar[r] & 1 \\
   1 \ar[r] & T' \ar[r] & W(1)' \ar[r] & W \ar[r] & 1 } \]
\end{itemize}
\end{frame}

\begin{frame}{\dots and how I came to it}
   \pause
   \begin{theorem*}[\alt<7->{Tits}{S.}]\label{thm:tits}
   The category $\mathcal{C}(W,S)$ has an initial object $V$ given by
   \[ \xymatrix{ 1 \ar[r] & \Z[\mathfrak{H}] \ar[r] & V \ar[r] & W \ar[r] & 1 } \]
   \pause where $V = \Z[\mathfrak{H}]\times W$ with product
   \[ (t,w)\cdot{}(t',w') = (tw(t')\X(w,w'), ww') \]
   \pause Given $W(1)$ in $\mathcal{C}(W,S)$, the induced morphism $\phi: V \rightarrow W(1)$ is determined by
   \[ \pause\phi(t,w) = \phi(t)\cdot n(w) \quad \pause\text{and}\quad \phi(a_s) = n(s)^2 \]
   \end{theorem*}
\end{frame}

\begin{frame}{Schreier theory}
   \pause For every abelian $W$-module $T$\pause
   \[ H^2(W,T) \simeq \{ \xymatrix{ 1 \ar[r] & T \ar[r] & G \ar[r] & W \ar[r] & 1  } \}/_\sim \]
   \pause where to an extension \uncover<5->{(and a set-theoretic section $n$)}
   \begin{equation}\label{eq:seq1} \xymatrix{ 1 \ar[r] & T \ar[r] & G \ar[r] & W \ar[r] \alt<5->{\ar@/_1.2em/[l]^n}{\ar@[white]@/_1.2em/[l]^{\color{white}n}} & 1 } \tag{$\ast$} \end{equation}
   \pause[6] one associates the $2$-cocycle
   \[ \phi(g,g') = n(g)n(g')n(gg')^{-1} \]
   \pause In particular
   \[ \text{\eqref{eq:seq1} splits} \quad \Leftrightarrow\quad [\phi] = 0 \]
\end{frame}

\begin{frame}
   \begin{cor*}[of Tits' thm]
      Given an object
      \[ \xymatrix{ 1 \ar[r] & T \ar[r] & W(1) \ar[r] & W \ar[r] \ar@/_1.2em/[l]^n & 1 } \]
      of $\mathcal{C}(W,S)$\pause , the underlying extension of $T$ by $W$ is classified by
      \[ [h\circ \X] = h_\ast [\X] \]
   \end{cor*}
\end{frame}

\begin{frame}{$N(k)$ as an object of $\mathcal{C}(W,S)$}
   \pause\begin{itemize}
      \item<2-> $\Phi$ - root system of $(G,T)$
      \item<3-> $U_\alpha \leq G$ - root subgroup \pause ($\alpha \in \Phi$)
   \end{itemize}
   \pause Choose
   \begin{itemize}
      \item<5-> $\Delta \subseteq \Phi$ - simple roots
      \item<6-> $u_\alpha \in U_\alpha(k)-\{1\}$ for $\alpha \in \Delta$
   \end{itemize}
   \pause[7] Put
   \begin{itemize}
      \item<8-> $S := \{ s_\alpha : \alpha \in \Delta \}$ \uncover<9->{$\Rightarrow$ $(W,S)$ is Coxeter group}
      \item<10-> $\{n_\alpha\} := U_{-\alpha}(k)u_\alpha U_{-\alpha}(k)\ \cap\ N(k)$
   \end{itemize}

   \pause[11]
   \begin{lemma*}
      For all $\alpha, \beta \in \Delta$
      \[ \underbrace{n_\alpha n_\beta n_\alpha \dots}_{m(s_\alpha,s_\beta)} = \underbrace{n_\beta n_\alpha n_\beta \dots}_{m(s_\alpha, s_\beta)} \]
      \pause[12] (where $m(s,t) := \op{ord}(st)$)
   \end{lemma*}
\end{frame}

\begin{frame}{$N(k)$ as an object of $\mathcal{C}(W,S)$}
   \begin{cor*}
      There exists a unique map of sets $n: W \rightarrow N(k)$ s.t.
      \begin{align*} n(ww') & = n(w)n(w') \text{ if } \ell(ww') = \ell(w)+\ell(w') \\
         n(s_\alpha) & = n_\alpha \text{ if } \alpha \in \Delta
\end{align*}
   \end{cor*}
   \pause By Tits' theorem, extension \eqref{the-seq} gives
   \[ h: \Z[\mathfrak{H}] \longrightarrow T(k) \]
\end{frame}

\end{document}
