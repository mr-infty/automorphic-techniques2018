\documentclass[pdf]{beamer}
\mode<presentation>{}
%\usetheme{Rochester}
\usecolortheme{orchid}
%\usecolortheme{seahorse}
\setbeamertemplate{theorems}[ams style]

\usepackage[utf8]{inputenc}

\usepackage[sans]{dsfont}
\usepackage{amsmath}
\usepackage{csquotes}
\usepackage[all,cmtip]{xy}
\usepackage{eqparbox}
\usepackage{textcomp}
\usepackage{booktabs}
\usepackage{longtable}
\usepackage{tabu}
\usepackage{bm}
\usepackage{soul}

\newcommand{\op}[1]{\operatorname{#1}}
\newcommand{\bbf}[1]{\mathds{#1}}
\newcommand{\Z}{\bbf{Z}}
\newcommand{\F}{\bbf{F}}
\newcommand{\C}{\bbf{C}}
\newcommand{\X}{\bbf{X}}

\newtheorem{question}{Question}
\newtheorem{proposition}{Proposition}
\newtheorem{cor}{Corollary}
\newtheorem*{theorem*}{Theorem}
\newtheorem*{lemma*}{Lemma}
\newtheorem*{cor*}{Corollary}
\newtheorem*{def*}{Definition}
\newtheorem*{NB*}{N.B}
\newtheorem*{problem*}{Problem}
\newtheorem*{proposition*}{Proposition}
\newtheorem{conjecture}{Conjecture}

\title{The Cohomology of Root Data}
\author{Nicolas A. Schmidt}
\date{8 October 2018}

\begin{document}

%% title frame
\begin{frame}
   \titlepage
\end{frame}

\begin{frame}{What's an ``automorphic technique''?}
   \pause
   According to \textit{Kurt Eissler}:
  \pause
   \blockquote[]{
      \textins{\textit{Automorphic techniques} are} artistic activites which take place in a borderland of autoplastic and alloplastic attitudes toward reality.\uncover<4->{ A work of art \textelp{} is \textit{autoplastic} in so far as, like a dream or symptom, it serves the {\color<6->[rgb]{1,0,0}solution of an inner conflict} and the {\color<6->[rgb]{1,0,0}fulfillment of a wish};\uncover<5->{ it is simultaneously \textit{alloplastic}, however, since it modifies reality by the {\color<6->[rgb]{1,0,0}creation of something original and new.}}}
   }
   (Taken from H. Kohut, \textit{Forms and Transformations of Narcissism.} American Imago, 1967)
\end{frame}

\begin{frame}{The problem}
   \pause{}Given
   \begin{itemize}
      \item $G$ - connected split \alt<9->{almost-simple s.s.}{reductive} group $G$ over field $k$\pause
      \item $T$ - maximal split torus in $G$
   \end{itemize}
   \pause have extension
   \[ \xymatrix{ 1 \ar[r] & T(k) \ar[r] & N(k) \ar[r] & W \ar[r] & 1 } \]
   \pause where
   \begin{itemize}
      \item $N$ - normalizer of $T$\pause
      \item $W$ - Weyl group of $\Phi(G,T)$\pause{}, subgroup of $\op{Aut}(T) = \op{GL}_{\Z}(X_\ast(T))$
   \end{itemize}\pause
   \begin{question}\label{q:one}
   For which $G$ does this extension split?
   \end{question}
\end{frame}

\begin{frame}{Brief history of the problem}
   \pause\begin{itemize}
      \item[1966] Tits' article ``Normalisateurs de Tores I.'' appears. \pause Announces complete solution in ``Normalisateurs de Tores II.''\pause , which \textit{never appeared}.\pause 
      \item<-8>[1974] Curtis, Wiederhold, and Williams give answer for \textit{simple} groups over $k = \C$.\pause
      \item<-8>[2013--17] Gal't solves it case-by-case for $k = \overline{\F}_p$ in several articles.\pause
      \item<-8>[2017] Question is settled in general by Adams and He.\pause
      \item[2018] I attempt myself at it.\pause
   \end{itemize}
\end{frame}

\begin{frame}{\dots and how I came to it}
   \pause For every Coxeter group $(W,S)$, there exists a canonical $2$-cocycle\pause
   \[ \X: W \times W \longrightarrow \Z[\mathfrak{H}] \]
   \pause given by
   \[ \X(w,w') = \prod_{ H \in \mathfrak{H}(w,w')} a_H \]
   \pause where
   \begin{align*} \Z[\mathfrak{H}] & = \text{free (multiplicative) abelian group on symbols $a_H$} \\
      \uncover<6->{\mathfrak{H} & = \{ wsw^{-1} : w \in W,\ s \in S\} \pause \uncover<7->{= \text{``hyperplanes''}} \\}
      \uncover<8->{\mathfrak{H}(w,w') & = \{ H \in \mathfrak{H} : \text{\uncover<9->{$H$ separates both $1$ from $w$ and $w$ from $ww'$}}\}\\}
      \uncover<10->{ & = \text{hyperplanes crossed twice by $1-w-ww'$}\\}
      \uncover<11->{ & = \text{\textbf{the thing that makes your life difficult}}}
   \end{align*}
\end{frame}

\begin{frame}{\dots and how I came to it}
   \pause Given $(W,S)$, consider category $\mathcal{C}(W,S)$ with\pause
\begin{itemize}
   \item[objs] Extensions \uncover<4->{together with set-theoretic sections $n$}
      \[ \xymatrix{ 1 \ar[r] & T \ar[r] & W(1) \ar[r] & W \ar[r] \alt<4->{\ar@/_1.2em/[l]^n}{\ar@[white]@/_1.2em/[l]^{\color{white}n}} & 1 } \]
   \pause[5]satisfying the ``braid relation''
   \[ \ell(ww') = \ell(w)+\ell(w') \quad \Rightarrow \quad n(ww') = n(w)n(w') \]
\item<6->[morphs] Commutative diagrams\uncover<7->{ preserving sections \uncover<8->{($h \circ n = n'$)}}
      \[ \xymatrix{ 1 \ar[r] & T \ar[d] \ar[r] & W(1) \ar[d]^h \ar[r] & W \ar[d]^{\op{id}} \ar[r] & 1 \\
   1 \ar[r] & T' \ar[r] & W(1)' \ar[r] & W \ar[r] & 1 } \]
\end{itemize}
\end{frame}

\begin{frame}{\dots and how I came to it}
   \pause
   \begin{theorem*}[\alt<7->{Tits}{S.}]\label{thm:tits}
   The category $\mathcal{C}(W,S)$ has an initial object $V$ given by
   \[ \xymatrix{ 1 \ar[r] & \Z[\mathfrak{H}] \ar[r] & V \ar[r] & W \ar[r] & 1 } \]
   \pause where $V = \Z[\mathfrak{H}]\times W$ with product
   \[ (t,w)\cdot{}(t',w') = (tw(t')\X(w,w'), ww') \]
   \pause Given $W(1)$ in $\mathcal{C}(W,S)$, the induced morphism $h: V \rightarrow W(1)$ is determined by
   \[ \pause h(t,w) = h(t)\cdot n(w) \quad \pause\text{and}\quad h(a_s) = n(s)^2 \]
   \end{theorem*}
\end{frame}

\begin{frame}{Schreier theory}
   \pause For every abelian $W$-module $T$\pause
   \[ H^2(W,T) \simeq \{ \xymatrix{ 1 \ar[r] & T \ar[r] & G \ar[r] & W \ar[r] & 1  } \}/_\sim \]
   \pause where to an extension \uncover<5->{(and a set-theoretic section $n$)}
   \begin{equation}\label{eq:seq1} \xymatrix{ 1 \ar[r] & T \ar[r] & G \ar[r] & W \ar[r] \alt<5->{\ar@/_1.2em/[l]^n}{\ar@[white]@/_1.2em/[l]^{\color{white}n}} & 1 } \tag{$\ast$} \end{equation}
   \pause[6] one associates the $2$-cocycle
   \[ \phi(g,g') = n(g)n(g')n(gg')^{-1} \]
   \pause In particular
   \[ \text{\eqref{eq:seq1} splits} \quad \Leftrightarrow\quad [\phi] = 0 \]
\end{frame}

\begin{frame}{Schreier theory}
   \begin{cor*}[of Tits' thm]
      Given an object
      \[ \xymatrix{ 1 \ar[r] & T \ar[r] & W(1) \ar[r] & W \ar[r] \ar@/_1.2em/[l]^n & 1 } \]
      of $\mathcal{C}(W,S)$\pause , the underlying extension of $T$ by $W$ is classified by\pause
      \[ [h\circ \X] = h_\ast [\X] \]
      \pause where $h: \Z[\mathfrak{H}] \rightarrow T$ denotes the induced map\pause , determined by
      \[ h(a_s) = n(s)^2 \]
   \end{cor*}
\end{frame}

\begin{frame}{$N(k)$ as an object of $\mathcal{C}(W,S)$}
   \pause Let
   \begin{itemize}
      \item<2-> $\Phi$ - root system of $(G,T)$
      \item<3-> $U_\alpha \leq G$ - root subgroup \pause ($\alpha \in \Phi$)
   \end{itemize}
   \pause Choose
   \begin{itemize}
      \item<5-> $\Delta \subseteq \Phi$ - simple roots
      \item<6-> $u_\alpha \in U_\alpha(k)-\{1\}$ for $\alpha \in \Delta$
   \end{itemize}
   \pause[7] Put
   \begin{itemize}
      \item<8-> $S := \{ s_\alpha : \alpha \in \Delta \}$ \uncover<9->{$\Rightarrow$ $(W,S)$ is Coxeter group}
      \item<10-> $\{n_\alpha\} := U_{-\alpha}(k)u_\alpha U_{-\alpha}(k)\ \cap\ N(k)$
   \end{itemize}
\end{frame}

\begin{frame}{$N(k)$ as an object of $\mathcal{C}(W,S)$}
   \begin{lemma*}
      For all $\alpha, \beta \in \Delta$
      \[ \underbrace{n_\alpha n_\beta n_\alpha \dots}_{m(s_\alpha,s_\beta)} = \underbrace{n_\beta n_\alpha n_\beta \dots}_{m(s_\alpha, s_\beta)} \]
      \pause[2] (where $m(s,t) := \op{ord}(st)$)
   \end{lemma*}
   \pause[3]
   \begin{cor*}
      There exists a unique map of sets $n: W \rightarrow N(k)$ s.t.
      \begin{align*} n(ww') & = n(w)n(w') \text{ if } \ell(ww') = \ell(w)+\ell(w') \\
         n(s_\alpha) & = n_\alpha \text{ if } \alpha \in \Delta
\end{align*}
   \end{cor*}
\end{frame}

\begin{frame}{Cohomological Reformulation of the Problem}
\begin{cor*}
   As an extension of $T(k)$ by $W$, $N(k)$ is classified by
   \[ h_\ast[\X] \in H^2(W,T(k)) \]
   \pause where $h:\Z[\mathfrak{H}] \rightarrow T(k)$ is determined by
   \[ h(a_{s_\alpha}) = n_\alpha^2 \quad (\alpha \in \Delta) \]
\end{cor*}
\pause In particular, question \ref{q:one} is equivalent to\pause
\begin{question}\label{q:two}
   When does
   \[ h_\ast[\X] = 0\]
   hold true?
\end{question}
\end{frame}

\begin{frame}{Getting rid of $G$}
   \pause \begin{itemize}
      \item<2-> $(X,\Phi,X^\vee,\Phi^\vee) := (X^\ast(T),\Phi,X_\ast(T),\Phi^\vee)$ - root datum
      \item<3-> $T(k) \simeq \op{Hom}_\Z(X,k^\times) \uncover<4->{\simeq X^\vee \otimes k^\times}$
   \end{itemize}
   \pause[5]
   \begin{lemma*}
      For all $\alpha \in \Delta$\uncover<6->{ (note $\alpha^\vee \in X_\ast(T)$ is a map $\bbf{G}_m \rightarrow T$)}
      \[ n_\alpha^2 = \alpha^\vee(k)(-1)\enskip \in\enskip T(k) \]
      \pause[7] i.e.
      \[ n_\alpha^2 = \alpha^\vee \otimes \only<8->{\eqmakebox[tauBOX][r]{$\tau^{-1}$}}\only<-7>{\eqmakebox[tauBOX][r]{$-1$}}\enskip \in\enskip \ X^\vee \otimes k^\times \]
   \end{lemma*}
   \pause[9] So $h:\Z[\mathfrak{H}] \rightarrow T(k) \uncover<10->{= X^\vee \otimes k^\times}$ \pause[11]is determined by\pause
   \[ h(a_{s_\alpha}) = \alpha^\vee \otimes \tau^{-1} \]
\end{frame}

\begin{frame}{Getting rid of $G$}
   \pause \begin{itemize}
      \item<2-> $\Phi \subseteq V$ - reduced irreducible root system
      \item<3-> $Q^\vee = \Z\left<\Phi^\vee\right> \leq V^\vee$ - coroot lattice
      \item<4-> $P^\vee = \{ \lambda \in V^\vee : \lambda(\Phi) \subseteq \Z \} \leq V^\vee$ - coweight lattice 
      \item<5-> $Q^\vee \subseteq X^\vee \subseteq P^\vee$ - sublattice\uncover<6->{ (automatically $W$-invariant)}
      \item<7-> $\Rightarrow (X,\Phi,X^\vee,\Phi^\vee)$ is the root datum of a connected split almost-simple s.s. group, and every such root datum is obtained in this way.
   \end{itemize}
   \pause[8] Thus question \ref{q:two} is equivalent to\pause
   \begin{question}\label{q:three}
      For which $\Phi$, $X^\vee$ \only<-9>{\eqmakebox[kBOX][l]{and $k$}}\only<10->{\eqmakebox[kBOX][l]{\textbf{and $\mathbf{k}$}}} does\uncover<11->{ (with $h(a_{s_\alpha}) = \alpha^\vee \otimes \tau^{-1}$)}
      \[ h_\ast[\X] = 0 \]
      hold true in $H^2(W,X^\vee\otimes k^\times)$?
   \end{question}
\end{frame}

\begin{frame}{Getting rid of $k$}
   \begin{lemma*}
      There exists a (unique) $W$-equivariant $\Z$-linear map
      \[ h_u: \Z[\mathfrak{H}] \longrightarrow X^\vee \otimes \F_2 \]
      satisfying
      \[ h_u(a_{s_\alpha}) = \alpha^\vee \otimes 1\quad \forall \alpha \in \Phi \]
      \pause Moreover, $h$ factors as
      \[ \Z[\mathfrak{H}] \stackrel{h_u}{\longrightarrow} X^\vee \otimes \F_2 \stackrel{\op{id}\otimes \eta_k}{\longrightarrow} X^\vee \otimes k^\times \]
      \pause where $\eta_k: \F_2 \twoheadrightarrow \mu_2(k) \subseteq k^\times$ is the obvious map \pause ($\eta_k(1) = \tau^{-1}$).
   \end{lemma*}
\end{frame}

\begin{frame}{Getting rid of $k$}
   \begin{def*}
      \begin{center}$\phi_u := h_u \circ \X \pause\enskip \in \enskip Z^2(W,X^\vee \otimes \F_2)$\end{center}
   \end{def*}
   \pause \begin{cor*}
      \begin{center}$h_\ast[\X] = (\op{id}\otimes \eta_k)_\ast [\phi_u]$\end{center}
   \end{cor*}
   \pause Thus, question \ref{q:three} is equivalent to \pause
   \begin{question}\label{q:four}
      For which $\Phi$, $X^\vee$ and $k$ does $[\phi_u]$ lie in the kernel of 
      \[ H^2(W,X^\vee \otimes \F_2) \stackrel{(\op{id}\otimes \eta_k)_\ast}{\longrightarrow} H^2(W,X^\vee\otimes k^\times) \]
      ?
   \end{question}
\end{frame}

\begin{frame}{Getting rid of $k$ completely}
   \pause \begin{NB*} If $-1 = 1$ in $k$, then $\phi_u = 0$. \pause So assume $-1 \neq 1$ from now on.
   \end{NB*}
   \pause Künneth's theorem applied to $\eta_k: \F_2 \hookrightarrow k^\times$ gives%
   \uncover<5->{\only<-5>{\[ \xymatrix{ H^2(W,X^\vee)\otimes \F_2 \ar[d]^{\op{id}\otimes \eta_k} \ar@^{(->}[r]^{\color{white}\iota} & H^2(W,X^\vee \otimes \F_2) \ar@{->>}[r] \ar[d]^{(\op{id}\otimes \eta_k)_\ast} & \op{Tor}^{\Z}_1(H^3(W,X^\vee),\F_2) \ar[d] \\
   H^2(W,X^\vee)\otimes k^\times \ar@^{(->}[r] & H^2(W,X^\vee \otimes k^\times) \ar@{->>}[r] & \op{Tor}^{\Z}_1(H^3(W,X^\vee),k^\times) } \]}%
   \only<6>{\[ \xymatrix{ H^2(W,X^\vee)\otimes \F_2 \ar[d]^{\op{id}\otimes \eta_k} \ar@^{(->}[r]^{\color{white}\iota} & H^2(W,X^\vee \otimes \F_2) \ar@{->>}[r] \ar[d]^{(\op{id}\otimes \eta_k)_\ast} & \op{Tor}^{\Z}_1(H^3(W,X^\vee),\F_2) \ar@^{(->}[d] \\
   H^2(W,X^\vee)\otimes k^\times \ar@^{(->}[r] & H^2(W,X^\vee \otimes k^\times) \ar@{->>}[r] & \op{Tor}^{\Z}_1(H^3(W,X^\vee),k^\times) } \]}%
   \only<7->{\[ \xymatrix{ H^2(W,X^\vee)\otimes \F_2 \ar[d]^{\op{id}\otimes \eta_k} \ar@^{(->}[r]^{\iota} & H^2(W,X^\vee \otimes \F_2) \ar@{->>}[r] \ar[d]^{(\op{id}\otimes \eta_k)_\ast} & \op{Tor}^{\Z}_1(H^3(W,X^\vee),\F_2) \ar@^{(->}[d] \\
   H^2(W,X^\vee)\otimes k^\times \ar@^{(->}[r] & H^2(W,X^\vee \otimes k^\times) \ar@{->>}[r] & \op{Tor}^{\Z}_1(H^3(W,X^\vee),k^\times) } \]}%
}
\pause[8] \begin{cor*}%
   \begin{center}$(\op{id}\otimes \eta_k)_\ast([\phi_u]) = 0 \quad \Leftrightarrow\pause\quad \exists x\enskip [\phi_u] = \iota(x) \pause\wedge (\op{id}\otimes \eta_k)(x) = 0$\end{center}
\end{cor*}
\end{frame}

\begin{frame}{Time to Compute}
   \pause \dots or is it? \pause Recall\pause
   \[ H^n(G,M) = H^n(\op{Hom}_{\Z[G]}(P_\bullet{},M)) \]
   \pause where
   \[ P_\bullet{}: \xymatrix{\dots \ar[r] & P_2 \ar[r]^{\partial_2} & P_1 \ar[r]^{\partial_1} & P_0 = \Z[G] \ar[r]^\varepsilon & \Z \ar[r] & 0 } \]
   is the standard (bar) resolution of $G$\pause , where
   \[ P_n = \bigoplus_{[g_1,\dots,g_n] \in G^{\times n}} \Z[G]\cdot{}[g_1,\dots,g_n] \]
   is the free $\Z[G]$-module over $G^{\times n}$. \pause Thus \uncover<9->{(for $M$ free)}
   \[ \op{Hom}_G(P_\bullet{},M)_n = \op{Hom}_{\Z[G]}(P_n,M) \uncover<8->{\simeq \op{Hom}_{\op{Set}}(G^{\times n}, M)} \]
   \pause[9] free $\Z$-module \pause of rank
   \[ (\# G)^n \cdot{} \op{rk}_\Z(M) \]
\end{frame}

\begin{frame}{A Problem}
   Weyl groups are \textit{big}. \pause For example
   \begin{itemize}
      \item<2->[$A_n$:] $W = S_{n+1}$ has order $(n+1)! \uncover<3->{\approx \sqrt{2\pi n}\left(\frac{n}{e}\right)^n}$
      \item<4->[$E_8$:] $W$ has order $2^{14}\cdot{}3^5\cdot{}5^2\cdot 7 \uncover<5->{ \approx 10^9}$
   \end{itemize}
   \pause[6] \begin{problem*}To e.g. compute $H^2(W,X^\vee)$ for $\Phi$ of type $E_8$, need to compute kernel of
   \[ 10^{27} \times 10^{18}\ \text{integer matrix} \]
   \end{problem*}
   \begin{itemize}
      \item<7-> $n_0 = 2.686 \cdot 10^{25}$ (Loschmidt constant) particles in $1 m^3$ of ideal gas (at $0 \text{\textdegree\ C}$ and $1 \op{atm}$)
      \item<8-> Default implementation in \texttt{Sage 8.2} can barely handle $100 \times 100$ matrices
   \end{itemize}
\end{frame}

\begin{frame}{A Solution}
   \pause The \textbf{DeConcini-Salvetti resolution} $\mathcal{CS}_\bullet$ \pause of a finite Coxeter group $(W,S)$ is given by
   \begin{align*}
      \mathcal{CS}_n & = \bigoplus_{\Gamma \in \mathcal{F}_n} \Z[W]\cdot{} [\Gamma] \\
      \uncover<4->{\mathcal{F}_n & = \{ \Gamma = (\Gamma_i)_{i \geq 1}\ :\ S\supseteq \Gamma_1 \supseteq \Gamma_2 \dots \text{ and $\# \Gamma = n$}\} \\}
      \uncover<5->{\# \Gamma & := \sum_{i \geq 1} \# \Gamma_i}
   \end{align*}
   \pause[6]
   \begin{itemize}
      \item<6-> $\# \mathcal{F}_n = \binom{\# S + n - 1}{n}\uncover<7->{ = O((\# S)^n)}$
      \item<8-> For $E_8$: $\mathcal{F}_2 = 36$, $\mathcal{F}_3 = 120$
   \end{itemize}
   \pause[9]\begin{NB*}To compute $H^2(W,X^\vee)$ for $E_8$ using $\mathcal{CS}_\bullet$, only need to compute kernel of
      \[ 960\times 288 \text{ integer matrix}\]
   \end{NB*}
\end{frame}

\begin{frame}{An Inner Conflict}
\pause \begin{problem*}Can only compute a \textit{finite} number of cases, but there are \textit{infinite} families $A_\ell, B_\ell, C_\ell, D_\ell$ of irreducible root systems.\end{problem*}
\pause \begin{center}So what's the point?\end{center}
\end{frame}

\begin{frame}{An Observation}
   \pause Consider
   \[ d(\ell) := \op{dim}_{\F_2} H^2(W_\ell,Q^\vee_\ell\otimes \F_2) \]
   \pause
   {\setlength{\tabulinesep}{5pt}
   \begin{longtabu}spread 10em{*8{X[-1,L,$$]}}
      \toprule
      \rowfont{\bf}
      \bm{\ell} & A & B & C & D \\
      \midrule
      \endhead
      1 & \uncover<14->{1} & \uncover<4->{\--} & \uncover<12->{\--} & \uncover<13->{\--} \\
      2 & \uncover<14->{0} & \uncover<5->{3} & \uncover<12->{3} & \uncover<13->{\--} \\
      3 & \uncover<14->{2} & \uncover<6->{5} & \uncover<12->{6} & \uncover<13->{2} \\
      4 & \uncover<14->{0} & \uncover<7->{4} & \uncover<12->{7} & \uncover<13->{8} \\
      5 & \uncover<14->{3} & \uncover<8->{7} & \uncover<12->{8} & \uncover<13->{4} \\
      6 & \uncover<14->{0} & \uncover<9->{4} & \uncover<12->{8} & \uncover<13->{8} \\
      7 & \uncover<14->{3} & \uncover<10->{7} & \uncover<12->{8} & \uncover<13->{4} \\
      8 & \uncover<14->{0} & \uncover<11->{4} & \uncover<12->{8} & \uncover<13->{8}
      \endline
   \end{longtabu}}
\end{frame}

\begin{frame}{Coincidence?}
\end{frame}

\begin{frame}{Coincidence? No, Representation stability!}
   \pause Note: $S_1 \subseteq S_2 \subseteq S_3 \subseteq S_4 \subseteq \dots$
   \pause\begin{theorem*}[Nakaoka '60]
      \begin{center}$\op{res}:H^k(S_{n+1},\F_p) \stackrel{\sim}{\longrightarrow} H^k(S_n,\F_p)\quad\text{ for $n \gg 0$}$\end{center}
   \end{theorem*}
   \pause \uncover<4->{\only<-4>{\eqmakebox[strikeBOX]{Generalization:}}\only<5->{\eqmakebox[strikeBOX]{\st{Generalization:}}}}
   \begin{theorem*}[Nagpal-Snowden '18]
      For every finitely generated $\mathbf{FI}$-module $M$ over $\F_p$
      \begin{center}$\op{dim}_{\F_p} H^k(S_n,M_n) = \op{dim}_{\F_p} H^k(S_{n+q}, M_{n+q}) \quad\text{ for $n \gg 0$}$\end{center}
      for some $q = p^r$.
   \end{theorem*}
\end{frame}

\begin{frame}{Another Problem}
   \begin{itemize}
      \item<1-> To decide question \ref{q:four}, need to know
      \[ \iota: H^2(W,X^\vee)\otimes \F_2 \hookrightarrow H^2(W,X^\vee\otimes \F_2) \]
      \pause[2] as well as $[\phi_u] \in H^2(W,X^\vee\otimes \F_2)$
   \item<3-> The theorem of Nagpal-Snowden only gives isomorphism types
   \end{itemize}
\end{frame}

\begin{frame}{\dots The Plot Thickens}
   Consider $H^2(W_\ell,Q^\vee_\ell\otimes \F_2)$ for the family $B_\ell$ again:\pause
   {\setlength{\tabulinesep}{5pt}
   \begin{longtabu}spread 10em{*3{X[-1,L,$$]}}
      \toprule
      \rowfont{\bf}
      \bm{\ell} & d(\ell) & \uncover<3->{[\bm{\phi}_u]} \\
      \midrule
      \endhead
      2 & 3 & \uncover<3->{(1,1,0)} \\
      3 & 5 & \uncover<3->{(1,1,1,0,1)} \\
      4 & 4 & \uncover<3->{(1,1,1,0)} \\
      5 & 7 & \uncover<3->{(1,1,0,1,0,1,0)} \\
      6 & 4 & \uncover<3->{(1,1,0,1)} \\
      7 & 7 & \uncover<3->{(1,1,0,1,0,1,0)} \\
      8 & 4 & \uncover<3->{(1,1,0,1)} \\
      \endline
   \end{longtabu}}
\end{frame}

\begin{frame}{\dots The Plot Thickens}
   Let $A_\ell$ denote the matrix of
   \[ \iota: H^2(W_\ell,Q^\vee_\ell) \otimes \F_2 \hookrightarrow H^2(W_\ell,Q^\vee_\ell\otimes \F_2) \]
   w.r.t. the computed bases.
   \pause \begin{center}
      {\setlength{\tabulinesep}{3pt}\begin{tabu}spread 1cm {>{\bf}X[-1,R,$$]*7{X[-1,C,$$]}}
         \toprule
         \bm{\ell} & 2 & 3 & 4 & 5 & 6 & 7 & 8 \\
         \bm{d(\ell)} & 3 & 5 & 4 & 7 & 4 & 7 & 4 \\
         \bm{A_\ell} & \uncover<3->{\begin{pmatrix} 0 \\ 1 \\ 1 \end{pmatrix}} & \uncover<4->{\begin{pmatrix} 0 & 1 \\ 0 & 0 \\ 0 & 0 \\ 1 & 0 \\ 0 & 0 \end{pmatrix}} & \uncover<5->{\begin{pmatrix} 1 \\ 0 \\ 0 \\ 1 \end{pmatrix}} & \uncover<6->{\begin{pmatrix} 0 & 1 \\ 0 & 0 \\ 0 & 0 \\ 0 & 0 \\ 1 & 0 \\ 0 & 0 \\ 0 & 0 \end{pmatrix}} & \uncover<7->{\begin{pmatrix} 1 \\ 0 \\ 0 \\ 0 \end{pmatrix}} & \uncover<8->{\begin{pmatrix} 0 & 1 \\ 0 & 0 \\ 0 & 0 \\ 0 & 0 \\ 1 & 0 \\ 0 & 0 \\ 0 & 0 \end{pmatrix}} & \uncover<9->{\begin{pmatrix} 1 \\ 0 \\ 0 \\ 0 \end{pmatrix}} \\
         \bottomrule
      \end{tabu}}
   \end{center}
\end{frame}

\begin{frame}{\dots The Plot Thickens}
   \begin{itemize}
      \item<2-> For $X \in \{A,B,C,D\}$: canonically $W_\ell \subseteq W_{\ell+1}$ and $Q_\ell \subseteq Q_{\ell+1}$
      \item<3-> Induces \textit{restriction}: $\op{res}: H^k(W_{\ell+1},Q^\vee_{\ell+1}) \rightarrow H^k(W_\ell,Q^\vee_\ell)$
   \end{itemize}
   \pause[4]
   \begin{lemma*}
   \begin{center}$\xymatrixcolsep{2pt}\xymatrix{ H^2(W_{\ell+1},Q^\vee_{\ell + 1})\otimes \F_2 \ar[d]^{\op{res}\otimes \op{id}} \ar@^{(->}[rrr]^\iota &&& H^2(W_{\ell + 1},Q^\vee_{\ell + 1}\otimes \F_2) \ar[d]^{\op{res}} & \ni [\phi_u] \ar@{|->}[d] \\
         H^2(W_\ell, Q^\vee_\ell) \otimes \F_2 \ar@{^(->}[rrr]^\iota &&& H^2(W_\ell, Q^\vee_\ell \otimes \F_2) & \ni [\phi_u]}$\end{center}
   \end{lemma*}
\end{frame}

\begin{frame}{A Conjecture}
   \begin{conjecture}
      For $\ell \gg 0$, restriction induces isomorphisms
      \begin{align*} \uncover<2->{H^2(W_{\ell+q},Q^\vee_{\ell + q}\otimes \F_2) & \stackrel{\sim}{\longrightarrow} H^2(W_\ell,Q^\vee_\ell \otimes \F_2) \\[.8em]}
         \uncover<3->{H^2(W_{\ell + q},Q^\vee_{\ell+q}) & \stackrel{\sim}{\longrightarrow} H^2(W_\ell,Q^\vee_\ell)}
      \end{align*}
      \pause[4] where $q = 2$ for types $A,B,D$ \pause and $q = 1$ for type $C$.
   \end{conjecture}
\end{frame}

\begin{frame}{A result}
   \begin{proposition*}[S.]The sequence
      \begin{center}$Q^\vee_1 \subseteq Q^\vee_2 \subseteq Q^\vee_3 \subseteq \dots$\end{center}
      forms an $\mathbf{FI}_X$-module $Q^\vee$. \pause For type $A$, it is \textit{finitely presented} \pause with generators in degrees $\leq 2$ \pause and relations in degrees $\leq 2$.
   \end{proposition*}
   \pause (Effective version of Nagpal-Snowden + Computations) $\Rightarrow$
   \begin{theorem*}[S.]
      For $k = 1,2,3$ and $\ell$ even, the group
      \begin{center}$H^k(S_{\ell+1},Q^\vee_\ell)$\end{center}
      is $2$-torsion free.
   \end{theorem*}
\end{frame}

\begin{frame}{}
   \begin{center}\Huge \textbf{Thanks!}\end{center}
\end{frame}

\begin{frame}{References}
   \begin{itemize}
      \item Tits, ``\textit{Normalisateurs de tores I.}'', J. Algebra (1966)
      \item Adams, He, ``\textit{Lifting elements of Weyl groups}'', J. Algebra (2017)
      \item Nagpal, Snowden, ``\textit{Periodicity in the cohomology of symmetric groups via divided powers}'', Proc. Lond. Math. Soc (2018)
      \item Wilson, ``\textit{$\mathbf{FI}_{\mathcal{W}}$-modules and stability criteria for representations of classical Weyl groups}'', Journal of Algebra (2014)
      \item \texttt{https://github.com/mr-infty/crd}
   \end{itemize}
\end{frame}
\end{document}
