\documentclass[pdf]{beamer}
\mode<presentation>{}

\usepackage[sans]{dsfont}
\usepackage{amsmath}
\usepackage{csquotes}

\newcommand{\op}[1]{\operatorname{#1}}
\newcommand{\bbf}[1]{\mathds{#1}}
\newcommand{\Z}{\bbf{Z}}

\title{The Cohomology of Root Data}
\author{Nicolas A. Schmidt}
\date{8 October 2018}

\begin{document}

%% title frame
\begin{frame}
   \titlepage
\end{frame}

\begin{frame}{What's an ``automorphic technique''?}
   \pause
   According to \textit{Kurt Eissler}:
  \pause
   \blockquote[]{
      \textins{\textit{Automorphic techniques} are} artistic activites which take place in a borderland of autoplastic and alloplastic attitudes toward reality. A work of art \textelp{} is \textit{autoplastic} in so far as, like a dream or symptom, it serves the {\color<4->[rgb]{1,0,0}solution of an inner conflict and the fulfillment of a wish}; it is simultaneously \textit{alloplastic}, however, since it modifies reality by the {\color<4->[rgb]{1,0,0}creation of something original and new}.
   }
   (Taken from H. Kohut, \textit{Forms and Transformations of Narcicissm.} American Imago, 1967)
\end{frame}

\begin{frame}{The problem}
\end{frame}

\end{document}
